\documentclass[11pt]{report}

\begin{document}
\title{My Tutorial}
\author{Mohona Datta}
\maketitle
\tableofcontents
\chapter{Introduction}
My first LaTeX document!
This is a line in text mode. 
All commands start with a backslash.
documentclass is the preamble, this the the body. 
Start of by declaring "title{•}" and "author{•}". Then "maketitle" (blackslash removed) It will display on a separate page in all instances apart from the article class. 
It is a good idea to compile as you go along.
A blank line of code will generate a new line (hard return), or two backslashes (soft return).\\
Now onto math mode, which needs to be embedded with a dollar sign:
Suppose we are given a rectangle with side lengths $(x+1)$ and $(x+3)$. Then the equation $A=x^2+4x+3$ represents the area of the rectangle. This is in-line math mode.\\
Display math mode is done by using double dollar signs.
$$A=x^2+4x+3$$
This will be on it's own line, and centre it.
\chapter{Common Maths Notation}
Superscript: 
$$2x^3$$ Using the exponent sign as per usual. Only the single character after the caret will show- $$2x^34$$ unless we wrap the entire indices with curly brackets: $$2x^{34}$$
$$2x^{3x^{4+5}}$$
\\Subscript:
Much the same as superscript, with underscore instead of a caret.
$$2x_3$$
$$2x_{34}$$
$$2x_{3x_{4+5}}$$
\\Greek Letters: Simply backslash and write the letter in maths mode
$$\pi$$
$$\alpha$$
$$A=\pi r^2$$
\\Trig Functions: slash tan/sin/cos
$$y=\sin(x)$$
$$y=\cos(x)$$
\end{document}
